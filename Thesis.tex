\documentclass[a4paper, oneside]{discothesis}

\usepackage[utf8]{inputenc}
\usepackage[T1]{fontenc}
\usepackage{graphicx}
\usepackage{float}
\usepackage{cprotect}
\usepackage{listings}
\usepackage{xcolor}
\usepackage{dirtree}
\usepackage[labelfont=bf,font=small,skip=5pt]{caption}
\usepackage[outdir=./figures/]{epstopdf}


\definecolor{codegreen}{rgb}{0,0.6,0}
\definecolor{codegray}{rgb}{0.5,0.5,0.5}
\definecolor{codepurple}{rgb}{0.58,0,0.82}
\definecolor{backcolour}{rgb}{0.95,0.95,0.92}


\lstdefinestyle{mystyle}{
	backgroundcolor=\color{backcolour},   
	commentstyle=\color{codegreen},
	keywordstyle=\color{magenta},
	numberstyle=\tiny\color{codegray},
	stringstyle=\color{codepurple},
	basicstyle=\ttfamily\footnotesize,
	breakatwhitespace=false,         
	breaklines=true,                 
	captionpos=b,                    
	keepspaces=true,                 
	numbers=left,                    
	numbersep=5pt,                  
	showspaces=false,                
	showstringspaces=false,
	showtabs=false,                  
	tabsize=2
}
\lstset{style=mystyle}


%%%%%%%%%%%%%%%%%%%%%%%%%%%%%%%%%%%%%%%%%%%%%%%%%%%%%%%%%%%%%%%%%%%%%%%%%%%%%%%%%%%%%%%%%%%%%%%%%
% DOCUMENT METADATA

\thesistype{Master's thesis} % Master's thesis, Bachelor's thesis, Semester thesis, Group Project
\title{On fast simulations of cardiac function}

\author{Diego Renner}
\email{drenner@student.ethz.ch}

\institute{Dep. of Mathematics \\[2pt]
ETH Zürich}

% Optionally, you can put in your own logo here

\supervisors{Prof. Dr. Siddhartha Mishra}

% Optionally, keywords and categories of the work can be shown (on the Abstract page)
%\keywords{Keywords go here.}
%\categories{ACM categories go here.}

\date{\today}

%%%%%%%%%%%%%%%%%%%%%%%%%%%%%%%%%%%%%%%%%%%%%%%%%%%%%%%%%%%%%%%%%%%%%%%%%%%%%%%%%%%%%%%%%%%%%%%%%

\begin{document}

\frontmatter % do not remove this line
\maketitle
\cleardoublepage

\begin{acknowledgements}
	
\end{acknowledgements}


\begin{abstract}
	
\end{abstract}

\tableofcontents

\mainmatter % do not remove this line

% Start writing here
\chapter{Introduction}
\chapter{Cardiovascular System}
This section serves as an overview of the cardiovascular system with a focus on the arteries of the systemic circulation. 
Section \ref{mbb} will describe the main building blocks of the cardiovascular system and their relation, section \ref{aw} will give the characteristics of the arterial walls, and section \ref{b} the characteristics of blood.
\section{Main Building Blocks} \label{mbb}
The cardiovascular system contains two major circulations.
These are the systemic and the pulmonary circulation. 
Both are fed with blood pumped from the heart.
Through the systemic circulation organs are supplied with oxygen-rich blood which is then transported back to the heart where it then enters the pulmonary circulation.
Through the pulmonary circulation the oxygen-poor blood is sent to the lungs where it is replenished with oxygen and then returned to the heart to reenter the systemic circulation.
Vessels carrying the blood away from the heart are referred to as arteries and the ones carrying the blood to the heart are referred to as veins.
In both the systemic and the pulmonary circulation the oxygen exchange (from blood to organs or from lungs to blood respectively) take place within so called capillary vessels.
Capillaries are very small vessels that connect arteries to veins.
From here on out our main focus will be on on the arteries of the systemic circulation. 
In the next two subsections we will describe the main characteristics of the arterial walls and the blood flowing within respectively.

\section{Arterial Wall} \label{aw}
An arterial wall consists of three layers or tunicae: an inner, middle, and outer layer also referred to as intima, media, and adventita respectively.
The tunica intima lines the inside of the artery. It is made up of a single layer of endothelial cells encased in a thin layer of elastin and collagen fibres (connective tissue). 
On the outside of the artery the tunica adventita is made up of connective tissue as well which attaches the artery to surounding connective tissue.
Inbetween these two layers lies the thickest layer, the tunica media, which consists of elastic fibres and muscle cells. 
The tunica media is responsible for the elasticity of the artery.
Arteries close to the heart have a tunica media that is mainly made up of elastic fibres and are referred to as elastic arteries while the tunica media of arteries farther away from the heart and of small arteries consist predominently of muscle cells and are called muscular arteris. 


\section{Blood} \label{b}

\chapter{1D-Model}
\chapter{Numerical Methods}
\chapter{Implementation}
\chapter{Sensitivity Analysis}
\chapter{Conclusion/ Future Work}


% This displays the bibliography for all cited external documents. All references have to be defined in the file references.bib and can then be cited from within this document.
\bibliographystyle{IEEEtran}
\bibliography{references/references}

% This creates an appendix chapter, comment if not needed.
\appendix
\chapter{Code Documentation/ Manual} 


\end{document}

